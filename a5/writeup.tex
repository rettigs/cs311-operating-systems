\documentclass[letterpaper,10pt,fleqn]{article}

%example of setting the fleqn parameter to the article class -- the below sets the offset from flush left (fl)
\setlength{\mathindent}{1cm}

\usepackage{graphicx}                                        

\usepackage{amssymb}                                         
\usepackage{amsmath}                                         
\usepackage{amsthm}                                          

\usepackage{alltt}                                           
\usepackage{float}
\usepackage{color}

\usepackage{balance}
\usepackage[TABBOTCAP, tight]{subfigure}
\usepackage{enumitem}

\usepackage{pstricks, pst-node}

%the following sets the geometry of the page
\usepackage{geometry}
\geometry{textheight=9in, textwidth=6.5in}

% random comment

\newcommand{\cred}[1]{{\color{red}#1}}
\newcommand{\cblue}[1]{{\color{blue}#1}}

\usepackage{hyperref}

\usepackage{textcomp}
\usepackage{listings}

\usepackage{wasysym}

\def\name{Sean Rettig}

%% The following metadata will show up in the PDF properties
\hypersetup{
  colorlinks = true,
  urlcolor = black,
  pdfauthor = {\name},
  pdfkeywords = {cs311 ``operating systems''},
  pdftitle = {CS 311 Project},
  pdfsubject = {CS 311 Project},
  pdfpagemode = UseNone
}

\parindent = 0.0 in
\parskip = 0.2 in

\pagestyle{empty}

\numberwithin{equation}{section}

\newcommand{\D}{\mathrm{d}}

\newcommand\invisiblesection[1]{%
  \refstepcounter{section}%
  \addcontentsline{toc}{section}{\protect\numberline{\thesection}#1}%
  \sectionmark{#1}}

\begin{document}

%to remove page numbers, set the page style to empty

\section*{Assignment 5 Write-up}
\hrule

\subsection*{README}
This README section exists to describe the proper usage of the programs so that they may be tested accordingly.

\subsubsection*{Usage}
This \LaTeX file and ip2asn.c may be compiled by running \texttt{make} with the provided makefile.  ip2asn.py and update.py should be executable with Python 2.6 or 2.7.

Usage information for each program may be viewed by running it with the -h flag, but will also be listed here for convenience:

\begin{verbatim}
Usage: ./ip2asn [-h] [-a ipaddress] [-p port] [-c infile] [-i indb] [-o outdb] [-d]...
    -h  view this help
    -a  specify the IP address to bind to, defaults to INADDR_ANY
    -p  specify the port to bind to, defaults to 54321
    -i  specify an input file to initialize the trie from
    -o  specify an output file to save the trie to upon termination
    -d  enable debug messages; use -dd for more even more messages

Usage: ./ip2asn.py [-h] [-a server_ip] [-p port] [-i infile] [-o outfile] [-q query_ip] [-d]...
    -h  view this help
    -a  specify the IP address of the server to connect to, defaults to 127.0.0.1
    -p  specify the port of the server to connect to, defaults to 54321
    -i  specify an input file of IP addresses to query the server with
    -o  specify an output file for query results, defaults to stdout; only works with -i
    -q  query a single IP address; cannot be used with -i
    -d  enable debug messages; use -dd for more even more messages

Usage: ./update.py [-h] [-a server_ip] [-p port] [-i infile] [-s] [-k] [-d]...
    -h  view this help
    -a  specify the IP address of the server to connect to, defaults to 127.0.0.1
    -p  specify the port of the server to connect to, defaults to 54321
    -i  specify an input file of CIDR prefixes and ASNs to update the server with
    -s  enable reporting mode; query server for usage statistics
    -k  kill the server
    -d  enable debug messages; use -dd for more even more messages
\end{verbatim}

For ip2asn.c and update.py, an infile takes the format of a newline-delimited list of (CIDR prefix, ASN) pairs (as in the provided entries1.txt and entries2.txt files).  This is also the format dumped into the outfile for ip2asn.c.  For ip2asn.py, an infile is a newline-delimited list of IPv4 addresses to query the server with (as in the provided queries.txt file), and an outfile will contain a newline-delimited list of (IP, ASN) pairs.

\subsubsection*{Examples}
A common usage of the ip2asn.c server program:
\begin{verbatim}
    ./ip2asn -i database -o database
\end{verbatim}
This allows the server to maintain its state between runs, as it initializes the trie from its previous dump.

A common usage of the update.c client program:
\begin{verbatim}
    ./update.py -i entries1.txt
    ./update.py -s
    ./update.py -si entries2.txt
    ./update.py -k
\end{verbatim}
The -i, -s, and -k flags may all be specified at once, in which case, the insertions happen first, followed by the statistics query, and finally the kill command.

Common usages of the ip2asn.py client program:
\begin{verbatim}
./ip2asn.py -q 1.2.3.4
./ip2asn.py -i queries.txt
./ip2asn.py -i queries.txt -o outfile
\end{verbatim}
The -q and -i flags may not be used together.

Additionally, all programs may be specified to bind/connect to a specific IP address and port.
\begin{verbatim}
./ip2asn -i database -o database -a 127.0.0.1 -p 33333
./update.py -sp 33333
./ip2asn.py -q 1.2.3.4 -a 128.193.37.168
\end{verbatim}
By default, ip2asn.c binds to INADDR\_ANY, while ip2asn.py and update.py connect to 127.0.0.1.  All three use port 54321 by default.

One may also view debugging messages by specifying the -d flag for any program, or -dd for even more verbose output.  Debug output provides an inside look on how the program is functioning.
\begin{verbatim}
./ip2asn -i database -o database -d
./ip2asn -i database -o database -dd
./update.py -di entries1.txt
\end{verbatim}

\subsection*{Design}
\begin{itemize}
    \item ip2asn.c will use the threaded version of the program from assignment 4 as its base, but rather than creating a predetermined number of workers which all read from stdin, they will be created whenever a new client connects to service that client, and will terminate when the client disconnects.  This worker-thread model is intended to satisfy the extra credit requirement.
    \item ip2asn.c will also be modified to record how many unique hosts have connected, how many queries have been answered, and how many prefixes are stored.  Update: Storing unique hosts was done via a hashtable, specified by uthash.h (http://troydhanson.github.io/uthash/).  This unique-host-tracking model is intended to satisfy the extra credit requirement.
    \item update.py and ip2asn.py will communicate with ip2asn.c via XML, with the commands taking the following forms (sans newlines and indentation):
    \begin{verbatim}
        <entry>
            <cidr> 1.2.3.4/5 </cidr>
            <asn> 6789 </asn>
        </entry>

        <query>
            <ip> 1.2.3.4 </ip>
        </query>

        <answer>
            <asn> 6789 </asn>
        </answer>

        <stats></stats>

        <stats>
            <hosts> 123 </hosts>
            <queries> 456 </queries>
            <prefixes> 789 </prefixes>
        </stats>

        <terminate></terminate>
    \end{verbatim}
    All XML commands will be terminated by newlines, and when a client is done sending commands, it will send a \begin{verbatim}<done></done>\end{verbatim} command.  Update: Since newlines are entered into the string whenever a packet is broken up, this method did not work.  Rather, I implemented basic character counting (on \(\langle\), \(\rangle\), and /) to parse the XML, allowing me to know when one command ended and the next began, ignoring newlines.  Additonally, I was able to simply read an EOF when the client closed the socket, so the ``done'' command was not necessary.
    \begin{itemize}
        \item The entry command is sent from update.py to ip2asn.c to add a new prefix/ASN pair to the trie.
        \item The query command is sent from ip2asn.py to ip2asn.c to get the ASN of the given IP address.
        \item The answer command is sent from ip2asn.c to ip2asb.py to return the ASN of the given IP address.
        \item The empty stats command is sent from update.py to ip2asn.c to query for usage statistics.
        \item The populated stats command is sent from ip2asn.c to update.py to return usage statistics.
        \item The terminate command is sent from update.py to ip2asn.c to terminate ip2asn.c; ip2asn.c will send itself a SIGINT signal via kill(), which causes it to dump the trie to the outfile (if one is specified) and terminate.
    \end{itemize}
\end{itemize}

\subsection*{Work Log}
\begin{verbatim}
commit fd0c85da4c522ac2b85af80b8b8049a9c5263957
Author: Sean Rettig <seanrettig@gmail.com>
Date:   Fri Mar 14 23:32:51 2014 -0700

    ip2asn.c: increase max workers

commit f0bc3f80f2e4bc5bff4cf5d0ae6b22376b2cabc1
Author: Sean Rettig <seanrettig@gmail.com>
Date:   Fri Mar 14 23:12:57 2014 -0700

    add testing files

commit 522a8ae08b9e3411ef12cced571cbc0b3a60533b
Author: Sean Rettig <seanrettig@gmail.com>
Date:   Fri Mar 14 23:10:48 2014 -0700

    writeup: add readme and usage information

commit 8d1771b62e8ab3e12c495862f3c7700ee8771a2a
Author: Sean Rettig <seanrettig@gmail.com>
Date:   Fri Mar 14 23:10:16 2014 -0700

    update.py: update help

commit 6b4517c747fd42e3303303c9aa0c3689daaf9215
Author: Sean Rettig <seanrettig@gmail.com>
Date:   Fri Mar 14 23:09:50 2014 -0700

    ip2asn.c: don't exit on nonexistent input file

commit fabadf18417452ea776967308f64272ae401acf2
Author: Sean Rettig <seanrettig@gmail.com>
Date:   Fri Mar 14 22:21:36 2014 -0700

    writeup: add challenges overcome and extra command documentation

commit 22ed85648e8aed0cabd635784074bad2e045041f
Author: Sean Rettig <seanrettig@gmail.com>
Date:   Fri Mar 14 21:18:05 2014 -0700

    Updating design and answering questions in writeup

commit 67463162764b9badad73311254b0b0eb03a47f41
Author: Sean Rettig <seanrettig@gmail.com>
Date:   Fri Mar 14 20:43:17 2014 -0700

    ip2asn.c: actually fix newlines, and zero out buffers before using them

commit 4186746ca69866c8ffea121be612f1beea5d00a1
Author: Sean Rettig <seanrettig@gmail.com>
Date:   Fri Mar 14 19:52:02 2014 -0700

    ip2asn.c: fix ip and port representations

commit 40c20ffff43eb5619f1d5018fbf465c3950a8cdb
Author: Sean Rettig <seanrettig@gmail.com>
Date:   Fri Mar 14 19:39:12 2014 -0700

    I guess I can't parse self-closing tags after all

commit f1b9421aa9668e8ab65f90ce80a7a8908aacfcef
Author: Sean Rettig <seanrettig@gmail.com>
Date:   Fri Mar 14 19:15:19 2014 -0700

    ip2asn.c: fix char concat bug, initialize variables only once for performance

commit 08b6de08df3c4a0d1c8df517280ec4badafb7452
Author: Sean Rettig <seanrettig@gmail.com>
Date:   Fri Mar 14 18:43:05 2014 -0700

    fix slashes in CIDR prefixes getting parsed, fix newlines getting parsed, let self-closing tags get parsed again

commit 045952194a2da8e02adc2023c7b08851cb50d902
Author: Sean Rettig <seanrettig@gmail.com>
Date:   Fri Mar 14 18:19:00 2014 -0700

    ip2asn.c: fixing random data write bug

commit 041f47c5210495bc2d22fe8c9a68d014f6503cf9
Author: Sean Rettig <seanrettig@gmail.com>
Date:   Fri Mar 14 18:02:08 2014 -0700

    ip2asn.c: out format now matches in format, same file can be used for both, memory leak with insert/entry fixed

commit f55f8b35a405057ed535b6e8a89b30731e5899aa
Author: Sean Rettig <seanrettig@gmail.com>
Date:   Fri Mar 14 16:35:15 2014 -0700

    ip2asn.c: can now handle packet breakups and client crashes, and has fixed usage info formatting

commit 229ef5ca97af0ea55a0f94cc2dd0ceb227bf4bf8
Author: Sean Rettig <seanrettig@gmail.com>
Date:   Fri Mar 14 16:33:39 2014 -0700

    use spaces because it makes things easier to parse

commit d8f8e53018a94c7354a42efe80a0fa4a55947e5e
Author: Sean Rettig <seanrettig@gmail.com>
Date:   Fri Mar 14 16:30:57 2014 -0700

    Revert "use self-closing XML tags when ideal" (self-closing tags are harder to parse)
    
    This reverts commit b95d75af94665a67a968158c889658ea81b27b74.

commit 48ae17a063d8b94443664c17d44729f6887ba9ae
Author: Sean Rettig <seanrettig@gmail.com>
Date:   Fri Mar 14 01:13:07 2014 -0700

    adding field numbers to formatted strings for python 2.6 compatibility

commit 18d8f5c2303d0f630f8f6528a49deb3045970226
Author: Sean Rettig <seanrettig@gmail.com>
Date:   Fri Mar 14 00:53:09 2014 -0700

    update.py: mostly done

commit 577bc663fa30127f8c03632bb9d8809756a762a8
Author: Sean Rettig <seanrettig@gmail.com>
Date:   Fri Mar 14 00:52:37 2014 -0700

    ip2asn.py: exit with proper exit code

commit d3ce4d5c9f6ea1cd9707a1aa733d2ad653573c58
Author: Sean Rettig <seanrettig@gmail.com>
Date:   Fri Mar 14 00:52:18 2014 -0700

    ip2asn.c: fixing derp syntax error

commit 2ba9a968accafea5c0bfa6ec2801ddae51bc3e49
Author: Sean Rettig <seanrettig@gmail.com>
Date:   Fri Mar 14 00:47:42 2014 -0700

    adding uthash.h to repo

commit 03689ead3dc117638f6ac8fab2177233ded6af21
Author: Sean Rettig <seanrettig@gmail.com>
Date:   Thu Mar 13 23:47:51 2014 -0700

    begin update.py, add additional usage information to all programs for the -h flag

commit 60b863fb1b5c0b2fe5ad4a5f8293029511b9d39c
Author: Sean Rettig <seanrettig@gmail.com>
Date:   Thu Mar 13 23:22:23 2014 -0700

    makes threads terminate once the client is done, improve output formatting, add debug message on invalid command

commit aa19807c5e04bc0c96f890f6cfdd41bb921de941
Author: Sean Rettig <seanrettig@gmail.com>
Date:   Thu Mar 13 21:37:37 2014 -0700

    ip2asn.c and ip2asn.py are now working properly

commit dbc7e5a7408a10c842795e88bb0a66ff55e54c36
Author: Sean Rettig <seanrettig@gmail.com>
Date:   Thu Mar 13 05:58:06 2014 -0700

    done with initial creation of ip2asn.py, still needs testing

commit d04db14a7f83d6639176f9e47d32bfb660065bb3
Author: Sean Rettig <seanrettig@gmail.com>
Date:   Wed Mar 12 22:45:08 2014 -0700

    ip2asn.c: fixing some bugs that occur up to listening for clients; need to make clients to test further

commit a4da57672a64c95743641232e0c291dda0c9cf55
Author: Sean Rettig <seanrettig@gmail.com>
Date:   Wed Mar 12 22:12:04 2014 -0700

    fixing compile errors for ip2asn.c

commit f1a18d70c45d7e96376f5683226eb629bcb89ac1
Author: Sean Rettig <seanrettig@gmail.com>
Date:   Wed Mar 12 21:15:01 2014 -0700

    ip2asn.c theoretically done, still requires testing

commit 89c9022a8ede7ebfa8b537dd168726f5167f2b7d
Author: Sean Rettig <seanrettig@gmail.com>
Date:   Wed Mar 12 20:51:25 2014 -0700

    ip2asn.c now stores usage statistics

commit abbe588482e18bcdc8053894246d27d2254aaafc
Author: Sean Rettig <seanrettig@gmail.com>
Date:   Wed Mar 12 19:25:22 2014 -0700

    send self SIGINT instead of SIGSTOP

commit a9c823a9c94d05d7b0871905bdd1bb7f7c953bb7
Author: Sean Rettig <seanrettig@gmail.com>
Date:   Wed Mar 12 19:22:44 2014 -0700

    ip2asn XML handling framework implemented, still needs nitty-gritty

commit 34c6cf1471bc002abc39cf95c5c3720d101f08d1
Author: Sean Rettig <seanrettig@gmail.com>
Date:   Wed Mar 12 19:02:53 2014 -0700

    ip2asn.c has everything except XML parsing now

commit b95d75af94665a67a968158c889658ea81b27b74
Author: Sean Rettig <seanrettig@gmail.com>
Date:   Wed Mar 12 18:40:52 2014 -0700

    use self-closing XML tags when ideal

commit 1d74d8a8a23dabca3c777efef203f83b07116eb5
Author: Sean Rettig <seanrettig@gmail.com>
Date:   Wed Mar 12 12:20:27 2014 -0700

    done with initial design in writeup

commit d7ac34e4f87d311237f49724dd5c0a3358648368
Author: Sean Rettig <seanrettig@gmail.com>
Date:   Mon Mar 10 15:17:35 2014 -0700

    Initial commit of files I will be working with
\end{verbatim}
\subsection*{Challenges Overcame}
\begin{itemize}
    \item One challenge I overcame involved very confusing behaviour with the XML parser in ip2asn.c; when I simply added a new variable to hold the previous character, the line buffer started getting multiple characters added to it whenever it received only one; after much debugging, I discovered that I had been using strcat with a single char, rather than a string, causing it to add arbitrary data to the line buffer!  This was easily fixed by simply setting the appropriate position in the buffer to the wanted character rather than using strcat.
    \item Another challenge I overcame was mentioned earlier as an update of the design; I was originally delimiting XML commands with newlines, unaware that newlines could be automatically added in arbitrary positions in the string as part of the socket connection.  This required me to add additional logic to parse the XML so I could know whether we had a full command or not, and whether to wait for more characters.
    \item At one point, I realized that the method I had previously been using to dump the trie upon termination did not match what I had to use for input, and that having two different types of input files to specify was silly.  I ended up implementing bin2dec() and binary\_to\_prefix() in addition to dec2bin() and prefix\_to\_binary() to convert the binary prefixes I used to be dumping back into CIDR notation.  However, after implementing it, I noticed that the first line of the output had random, garbled data!  After a bit of debugging, I realized that I needed to zero out the buffer I was using to store the CIDR prefix before using it.
\end{itemize}

\subsection*{Questions}
\begin{enumerate}
    \item The main purpose of this assignment was to learn the usage and semantics of socket creation and I/O, with secondary purposes being to practice Python and parsing strings/XML.  This was also an exercise in problem solving skills, as this system is the most complex one we have written so far and has lots of minor little details to worry about as well as having the main architecture of the system be largely undefined, requiring us to design and implement it on our own.
    \item I ensured my solution was correct by testing each program option under various circumstances, such as when the trie is empty, when the trie already has entries, when new entries are being added, when duplicate entries are being added, when input/output files are specified, when multiple clients are connected, whether batch mode or one-off mode is used for queries, when operations (entries, queries, and stat queries) were interleaved and/or performed in different orders, and when multiple options are specified at once, such as -i, -s, and -k all at the same time on update.py.  The README section describes specifically describes tests that can be performed to ensure the system's correctness.
    \item Obviously, I learned a lot about sockets and Python, given that I had never used sockets before and knew almost no Python; I had to constantly look up even the most basic things about Python, such as how to define functions, how to print, how to format strings, how to catch exceptions, how to check if variables are null/None, how to parse strings using regular expressions, and how to perform file I/O.  Additionally, I learned about parsing my XML commands in a robust enough way to handle being transferred over a network.
\end{enumerate}

\end{document}
