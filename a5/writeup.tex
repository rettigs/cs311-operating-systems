\documentclass[letterpaper,10pt,fleqn]{article}

%example of setting the fleqn parameter to the article class -- the below sets the offset from flush left (fl)
\setlength{\mathindent}{1cm}

\usepackage{graphicx}                                        

\usepackage{amssymb}                                         
\usepackage{amsmath}                                         
\usepackage{amsthm}                                          

\usepackage{alltt}                                           
\usepackage{float}
\usepackage{color}

\usepackage{balance}
\usepackage[TABBOTCAP, tight]{subfigure}
\usepackage{enumitem}

\usepackage{pstricks, pst-node}

%the following sets the geometry of the page
\usepackage{geometry}
\geometry{textheight=9in, textwidth=6.5in}

% random comment

\newcommand{\cred}[1]{{\color{red}#1}}
\newcommand{\cblue}[1]{{\color{blue}#1}}

\usepackage{hyperref}

\usepackage{textcomp}
\usepackage{listings}

\usepackage{wasysym}

\def\name{Sean Rettig}

%% The following metadata will show up in the PDF properties
\hypersetup{
  colorlinks = true,
  urlcolor = black,
  pdfauthor = {\name},
  pdfkeywords = {cs311 ``operating systems''},
  pdftitle = {CS 311 Project},
  pdfsubject = {CS 311 Project},
  pdfpagemode = UseNone
}

\parindent = 0.0 in
\parskip = 0.2 in

\pagestyle{empty}

\numberwithin{equation}{section}

\newcommand{\D}{\mathrm{d}}

\newcommand\invisiblesection[1]{%
  \refstepcounter{section}%
  \addcontentsline{toc}{section}{\protect\numberline{\thesection}#1}%
  \sectionmark{#1}}

\begin{document}

%to remove page numbers, set the page style to empty

\section*{Assignment 5 Write-up}
\hrule

\subsection*{Design}
\begin{itemize}
    \item ip2asn.c will use the threaded version of the program from assignment 4 as its base, but rather than creating a predetermined number of workers which all read from stdin, they will be created whenever a new client connects to service that client, and will terminate when the client disconnects.
    \item ip2asn.c will also be modified to record how many unique hosts have connected, how many queries have been answered, and how many prefixes are stored.
    \item update.py and ip2asn.py will communicate with ip2asn.c via XML, with updates and queries taking the following forms (sans whitespace):
    \begin{verbatim}
        <entry>
            <cidr>1.2.3.4/5</cidr>
            <asn>6789</asn>
        </entry>

        <query>
            <ip>1.2.3.4</ip>
        </query>

        <answer>
            <asn>6789</asn>
        </answer>

        <stats />

        <stats>
            <hosts>123</hosts>
            <queries>456</queries>
            <prefixes>789</prefixes>
        </stats>

        <terminate />
    \end{verbatim}
\end{itemize}

\subsection*{Work Log}
\begin{verbatim}
\end{verbatim}
\subsection*{Challenges Overcame}
\begin{itemize}
    \item
\end{itemize}

\subsection*{Questions}
\begin{enumerate}
    \item
\end{enumerate}

\end{document}
