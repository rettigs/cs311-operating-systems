\documentclass[letterpaper,10pt,fleqn]{article}

%example of setting the fleqn parameter to the article class -- the below sets the offset from flush left (fl)
\setlength{\mathindent}{1cm}

\usepackage{graphicx}                                        

\usepackage{amssymb}                                         
\usepackage{amsmath}                                         
\usepackage{amsthm}                                          

\usepackage{alltt}                                           
\usepackage{float}
\usepackage{color}

\usepackage{balance}
\usepackage[TABBOTCAP, tight]{subfigure}
\usepackage{enumitem}

\usepackage{pstricks, pst-node}

%the following sets the geometry of the page
\usepackage{geometry}
\geometry{textheight=9in, textwidth=6.5in}

% random comment

\newcommand{\cred}[1]{{\color{red}#1}}
\newcommand{\cblue}[1]{{\color{blue}#1}}

\usepackage{hyperref}

\usepackage{textcomp}
\usepackage{listings}

\usepackage{wasysym}

\def\name{Sean Rettig}

%% The following metadata will show up in the PDF properties
\hypersetup{
  colorlinks = true,
  urlcolor = black,
  pdfauthor = {\name},
  pdfkeywords = {cs311 ``operating systems''},
  pdftitle = {CS 311 Project},
  pdfsubject = {CS 311 Project},
  pdfpagemode = UseNone
}

\parindent = 0.0 in
\parskip = 0.2 in

\pagestyle{empty}

\numberwithin{equation}{section}

\newcommand{\D}{\mathrm{d}}

\newcommand\invisiblesection[1]{%
  \refstepcounter{section}%
  \addcontentsline{toc}{section}{\protect\numberline{\thesection}#1}%
  \sectionmark{#1}}

\begin{document}

%to remove page numbers, set the page style to empty

\section*{Assignment 1}
\hrule

\begin{enumerate}
\item scp and rsync
\item Revision control systems allow for changes to a set of files to be tracked so that they can be easily reviewed, applied, reverted, etc.  You can create a git repo in your current directory with the command "git init".
\item Redirection allows a program's input and/or output to be read from and/or written to a file, respectively.  Piping allows one program to read the output of another program as input.
\item Make is tool that can be used to build programs automatically.  It is useful because it saves time and effort when developing and/or deploying software and helps prevent human error during the build process.
\item Makefiles consist of five different things:
\begin{itemize}
\item Explicit rules, which describe how to process specific targets.  Syntax:
\begin{verbatim}
targets : prerequisites
        recipe
        ...
\end{verbatim}
or
\begin{verbatim}
targets : prerequisites ; recipe
        recipe
        ...
\end{verbatim}
\item Implicit rules, which describe how to process targets based on file names.
\item Variable definitions, which store strings that can be used in the text later.
\item Directives, which instruct make to perform special actions.
\item Comments, which can be used to describe what is happening within the makefile. All text after a \# up to the end of the line is considered to be a comment. A literal \# can be escaped with a \, and a trailing \ after a comment lets it continue to the next line.
\end{itemize}
\item \begin{verbatim}find . -type f -exec file '{}' \;\end{verbatim} (from the man page)
\end{enumerate}

\end{document}
