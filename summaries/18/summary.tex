\documentclass[letterpaper,10pt,fleqn]{article}

%example of setting the fleqn parameter to the article class -- the below sets the offset from flush left (fl)
\setlength{\mathindent}{1cm}

\usepackage{graphicx}                                        

\usepackage{amssymb}                                         
\usepackage{amsmath}                                         
\usepackage{amsthm}                                          

\usepackage{alltt}                                           
\usepackage{float}
\usepackage{color}

\usepackage{balance}
\usepackage[TABBOTCAP, tight]{subfigure}
\usepackage{enumitem}

\usepackage{pstricks, pst-node}

%the following sets the geometry of the page
\usepackage{geometry}
\geometry{textheight=9in, textwidth=6.5in}

% random comment

\newcommand{\cred}[1]{{\color{red}#1}}
\newcommand{\cblue}[1]{{\color{blue}#1}}

\usepackage{hyperref}

\usepackage{textcomp}
\usepackage{listings}

\usepackage{wasysym}

\def\name{Sean Rettig}

%% The following metadata will show up in the PDF properties
\hypersetup{
  colorlinks = true,
  urlcolor = black,
  pdfauthor = {\name},
  pdfkeywords = {cs311 ``operating systems''},
  pdftitle = {CS 311 Project},
  pdfsubject = {CS 311 Project},
  pdfpagemode = UseNone
}

\parindent = 0.0 in
\parskip = 0.2 in

\pagestyle{empty}

\numberwithin{equation}{section}

\newcommand{\D}{\mathrm{d}}

\newcommand\invisiblesection[1]{%
  \refstepcounter{section}%
  \addcontentsline{toc}{section}{\protect\numberline{\thesection}#1}%
  \sectionmark{#1}}

\begin{document}

%to remove page numbers, set the page style to empty

\noindent {\large \bf Name: Sean Rettig \hfill Summary 18}

\noindent {\large \bf ID\#: 931-650-839 }

{\Large CS311 Operating Systems I}

\subsection*{TLPI Chapters 43, 51, and 53}

Chapter 43 introduces some of the basic considerations behind the various methods of IPC (InterProcess Communication), dividing them into three categories: communication (exchanging data between processes), synchronization (synchronizing actions between processes/threads), and signals, which fall in between the other two categories.  The communication category is further divided into memory sharing techniques (such as System V shared memory, POSIX shared memory, and memory mapping) and data transfer techniques (including psuedoterminals, messages, and byte streams, such as pipes, FIFOs, and sockets).  The synchronization category is further divided into semaphores, mutexes, condition variables, and file locks.  Signals also exist in two types: standard and realtime.  Discussed are the pros and cons of each with regards to their functionality, portability, accessibility, persistence, performance, and ease of use.

Chapter 51 expands on the POSIX-specific methods of IPC, including POSIX message queues (which allow messages to be passed between processes), POSIX semaphores (which control access to limited resources to synchronize multiple processes/threads), and POSIX shared memory (which allows processes to directly share regions of memory).  Discussed are the basic similarities between their use and also their differences compared to the corresponding System V IPC methods.

Chapter 53 focuses on POSIX semaphores in particular, including both named semaphores and unnamed semaphores.  Named semaphores can be created/opened using the sem\_open() function, and can be incremented or decremented using the sem\_post() and sem\_wait() functions, respectively.  The current value of a semaphore can be retrieved with the sem\_getvalue() function, and semaphores can be closed using sem\_close().  Once all processes have closed a semaphore, it can be marked for deletion with sem\_unlink().  Unnamed semaphores function similarly, but exist only in shared memory between processes and must be initialized and destroyed manually, using the sem\_init() and sem\_destroy() functions, respectively.  Also discussed are the differences of POSIX semaphores compared to System V semaphores and Pthreads mutexes.

\end{document}
