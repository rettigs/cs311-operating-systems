\documentclass[letterpaper,10pt,fleqn]{article}

%example of setting the fleqn parameter to the article class -- the below sets the offset from flush left (fl)
\setlength{\mathindent}{1cm}

\usepackage{graphicx}                                        

\usepackage{amssymb}                                         
\usepackage{amsmath}                                         
\usepackage{amsthm}                                          

\usepackage{alltt}                                           
\usepackage{float}
\usepackage{color}

\usepackage{balance}
\usepackage[TABBOTCAP, tight]{subfigure}
\usepackage{enumitem}

\usepackage{pstricks, pst-node}

%the following sets the geometry of the page
\usepackage{geometry}
\geometry{textheight=9in, textwidth=6.5in}

% random comment

\newcommand{\cred}[1]{{\color{red}#1}}
\newcommand{\cblue}[1]{{\color{blue}#1}}

\usepackage{hyperref}

\usepackage{textcomp}
\usepackage{listings}

\usepackage{wasysym}

\def\name{Sean Rettig}

%% The following metadata will show up in the PDF properties
\hypersetup{
  colorlinks = true,
  urlcolor = black,
  pdfauthor = {\name},
  pdfkeywords = {cs311 ``operating systems''},
  pdftitle = {CS 311 Project},
  pdfsubject = {CS 311 Project},
  pdfpagemode = UseNone
}

\parindent = 0.0 in
\parskip = 0.2 in

\pagestyle{empty}

\numberwithin{equation}{section}

\newcommand{\D}{\mathrm{d}}

\newcommand\invisiblesection[1]{%
  \refstepcounter{section}%
  \addcontentsline{toc}{section}{\protect\numberline{\thesection}#1}%
  \sectionmark{#1}}

\begin{document}

%to remove page numbers, set the page style to empty

\noindent {\large \bf Name: Sean Rettig \hfill Summary 20}

\noindent {\large \bf ID\#: 931-650-839 }

{\Large CS311 Operating Systems I}

\subsection*{TLPI Chapters 56 and 59}

Chapter 56 focuses on sockets, a method of IPC that allows applications to communicate locally or over a network.  Each application must open a socket using either a path (if local) or IP address and port number (if over a network).  Sockets are commonly implemented as stream sockets and datagram sockets, where the former provides reliable, connection-oriented data transfer, and the latter provides neither, but does preserve message boundaries.  Both implementations allow data to flow bidirectionally.

Sockets can be created using the socket() system call, and can be bound to another address with the bind() system call.  Applications can then use listen() to accept incoming connection requests, which can be accepted with accept().  Alternatively, an application can initiate a connection using connect().  For datagram sockets, recvfrom() and sendto() can be used to receive and send datagrams, respectively.

Chapter 59 expands on a particular type of sockets known as internet domain sockets, or sockets in the IPv4 (AF\_INET) and IPv6 (AF\_INET6) domains.  Discussed are various considerations to be made regarding network byte order, data representation, address representation, host and service conversions, and the differences between IPv4/IPv6 and UNIX domain sockets.

\end{document}
