\documentclass[letterpaper,10pt,fleqn]{article}

%example of setting the fleqn parameter to the article class -- the below sets the offset from flush left (fl)
\setlength{\mathindent}{1cm}

\usepackage{graphicx}                                        

\usepackage{amssymb}                                         
\usepackage{amsmath}                                         
\usepackage{amsthm}                                          

\usepackage{alltt}                                           
\usepackage{float}
\usepackage{color}

\usepackage{balance}
\usepackage[TABBOTCAP, tight]{subfigure}
\usepackage{enumitem}

\usepackage{pstricks, pst-node}

%the following sets the geometry of the page
\usepackage{geometry}
\geometry{textheight=9in, textwidth=6.5in}

% random comment

\newcommand{\cred}[1]{{\color{red}#1}}
\newcommand{\cblue}[1]{{\color{blue}#1}}

\usepackage{hyperref}

\usepackage{textcomp}
\usepackage{listings}

\usepackage{wasysym}

\def\name{Sean Rettig}

%% The following metadata will show up in the PDF properties
\hypersetup{
  colorlinks = true,
  urlcolor = black,
  pdfauthor = {\name},
  pdfkeywords = {cs311 ``operating systems''},
  pdftitle = {CS 311 Project},
  pdfsubject = {CS 311 Project},
  pdfpagemode = UseNone
}

\parindent = 0.0 in
\parskip = 0.2 in

\pagestyle{empty}

\numberwithin{equation}{section}

\newcommand{\D}{\mathrm{d}}

\newcommand\invisiblesection[1]{%
  \refstepcounter{section}%
  \addcontentsline{toc}{section}{\protect\numberline{\thesection}#1}%
  \sectionmark{#1}}

\begin{document}

%to remove page numbers, set the page style to empty

\section*{Assignment 2 Write-up}
\hrule

\subsection*{Design}
\begin{itemize}
    \item The 3 different types of processes will be called "reader", "scorer", and "combiner", with the main process eventually becoming the combiner after parsing the arguments, creating the pipes, and forking off the children.
    \item To implement the maximum of 8 tfidf processes at once, the combiner will first spawn 8, then spawn another one each time it gets an EOF from a tfidf process's pipe (and waits for it) as long as there are still additional files to search.
\end{itemize}

\subsection*{Work Log}
\begin{verbatim}

\end{verbatim}

\subsection*{Challenges Overcame}
\begin{itemize}
    \item Figuring out how to use opendir(), readdir(), and getline() was a little tricky at first, but I figured it out.
    \item Learning how to properly format sscanf() and snprintf() strings was confusing, but I got it after much trial and error.
    \item Learning how to recurse directories was difficult, but using a recursive function made it easier to deal with.
\end{itemize}

\subsection*{Questions}
\begin{enumerate}
    \item The main point of the assignment was to learn the basics of performing file I/O in C using the Linux programming interface.
    \item I tested all functions of bdiff with different files in different situations, including those where one file was bigger than the other and where multiple changes were made.  I tested bpatch in a similar way and achieved successful results, but was not able to get bpatch -r working properly; this was tested using several files in nested directories.
    \item I learned how to use opendir(), readdir(), sscanf(), snprintf(), and getline().
\end{enumerate}

\end{document}
