\documentclass[letterpaper,10pt,titlepage]{article}

\usepackage{graphicx}                                        

\usepackage{amssymb}                                         
\usepackage{amsmath}                                         
\usepackage{amsthm}                                          

\usepackage{alltt}                                           
\usepackage{float}
\usepackage{color}

\usepackage{url}

\usepackage{balance}
\usepackage[TABBOTCAP, tight]{subfigure}
\usepackage{enumitem}

\usepackage{pstricks, pst-node}

\usepackage{geometry}
\geometry{textheight=10in, textwidth=7.5in}

%random comment

\newcommand{\cred}[1]{{\color{red}#1}}
\newcommand{\cblue}[1]{{\color{blue}#1}}

\usepackage{hyperref}

\def\name{D. Kevin McGrath}

%pull in the necessary preamble matter for pygments output
\input{pygments.tex}

%% The following metadata will show up in the PDF properties
\hypersetup{
  colorlinks = true,
  urlcolor = black,
  pdfauthor = {\name},
  pdfkeywords = {cs311 ``operating systems'' files filesystem I/O},
  pdftitle = {CS 311 Project 1: UNIX File I/O},
  pdfsubject = {CS 311 Project 1},
  pdfpagemode = UseNone
}

\parindent = 0.0 in
\parskip = 0.2 in

\begin{document}

Below is some math.

$
\oint_0^{2\pi} x \log x \partial x
$

$$
\oint_0^2\pi x \log x \partial x
$$

\begin{equation}
\oint_0^{2\pi} x \log x \partial x
\end{equation}

Above is some more math.

If I want a literal underscore (like\_this), I have to escape it, using a backslash ($\backslash$).


\begin{itemize}
\item This is a bulleted list.
\item Look, another bullet.
\end{itemize}

\begin{enumerate}
\item This one is numbered.
\item Yay for \#2.
\item[new label] This one isn't actually numbered...
\end{enumerate}

\section*{Name of section}

This is langle and rangle in math mode: $\langle \rangle$

This is langle and rangle without math mode: $<>$

This is langle and rangle without escaping: <>

\begin{tabular}{|c|c|c|}
  \hline
  1 & 2 & 3 \\ \hline
  4 & 544432432 & 6 \\ \hline
  7 & 8 & 9 \\ \hline 
\end{tabular}

%to include an image:
%\includegraphics[width=\textwidth]{image.eps}


%input the pygmentized output of mt19937ar.c, using a (hopefully) unique name
%this file only exists at compile time. Feel free to change that.
%\input{__mt.h.tex}
\end{document}
