\documentclass[letterpaper,10pt,onecolumn,titlepage]{article}

\usepackage{graphicx}                                        
\usepackage{amssymb}                                         
\usepackage{amsmath}                                         
\usepackage{amsthm}                                          

\usepackage{alltt}                                           
\usepackage{float}
\usepackage{color}
\usepackage{url}

\usepackage{balance}
\usepackage[TABBOTCAP, tight]{subfigure}
\usepackage{enumitem}
\usepackage{pstricks, pst-node, pst-tree}


\usepackage{geometry}
\geometry{textheight=9.5in, textwidth=7in, bmargin=1in}

% random comment

\newcommand{\cred}[1]{{\color{red}#1}}
\newcommand{\cblue}[1]{{\color{blue}#1}}

\usepackage{hyperref}
\usepackage{geometry}

\def\name{D. Kevin McGrath}

\parindent = 0.0 in
\parskip = 0.1 in

%% The following metadata will show up in the PDF properties
\hypersetup{
  colorlinks = true,
  urlcolor = black,
  pdfauthor = {\name},
  pdfkeywords = {cs411 ``operating systems'' files filesystem I/O},
  pdftitle = {CS 411 Final},
  pdfsubject = {CS 411 Final},
  pdfpagemode = UseNone
}

\pagestyle{empty}
\begin{document}

\noindent {\large \bf Name: \line(1,0){200} \hfill Quiz 2}

\noindent {\large \bf ID\#: \line(1,0){200} }

{\Large CS311 Operating Systems I}

\emph{Please use this \LaTeX\ file as a template, and answer the following questions.
  Please be detailed in your answers.}

\emph{There are \ref{last} questions on this quiz. Please answer all of them clearly.
  Points are awarded for more complete answers. Brevity buys you nothing.}

\emph{You will be turning in the complete version of this file, plus a makefile to build
  it. Please ensure you fill out the name and ID fields above, as well.}

\begin{enumerate}%[itemsep=0.1 in]
\item What is the name of your TA?

\begin{quote}
Devlin Junker
\end{quote}

\item Describe the concept of scatter/gather I/O. How does one perform such I/O in C?
  What benefits or downsides does it have?

\begin{quote}
Scatter/gather I/O is a mechanism by which data can be read from or written to a file using multiple sequential buffers while remaining an atomic operation.  It can be performed using the readv(), writev(), preadv(), and/or pwritev() system calls.  The primary benefit of scatter/gather I/O is convenience and speed, as it eliminates the need to perform multiple I/O calls (or convert buffers with a single I/O call) in the event that multiple buffers need to be read/written.
\end{quote}

\item The third argument for \texttt{lseek} is \emph{whence}. What values can this
  parameter take on, and what does each mean?

\begin{quote}
\begin{itemize}
    \item \texttt{SEEK\_SET}: offset = \texttt{offset}
    \item \texttt{SEEK\_CUR}: offset = \texttt{offset} + current location
    \item \texttt{SEEK\_SET}: offset = \texttt{offset} + size of file
\end{itemize}
\end{quote}

\item Describe at least 3 ways to time your code. What are the positives and negatives of
  each method?

\begin{quote}
\begin{itemize}
    \item Using gettimeofday() before and after: requires changing C code, only gives real time.
    \item Using times(): requires changing C code, only gives user and system time.
    \item Using the time function to run your program: simple, provides user, system, and real time.
\end{itemize}
\end{quote}

\item What system call is used to change the timestamp on a file? What restrictions are
  imposed on you when you use this function?

\begin{quote}
    utimensat(); File permissions impose restrictions on whether the timestamps can be modified.
\end{quote}

\item Describe the process of installing a signal handler using \texttt{sigaction},
  including any specific issues that need to be considered.

\begin{quote}

\end{quote}

\item Signals can be handled in 3 ways. Name them. What constants are defined for your
  use in setting these actions? Is one way better to use than others?

\begin{quote}
    Default action, ignore, or use signal handler.
\end{quote}

\item Give the definition of the \texttt{siginfo\_t} struct, and describe each of the
  fields.

\begin{quote}

\end{quote}

\item How do system calls and signals interact?

\begin{quote}
    System call interrupted by a signal generally fail with an EINTR error code, but can often be manually restarted or set to restart automatically using the \texttt{SA\_RESTART} flag.
\end{quote}

\item What are \emph{async-signal-safe} functions? Give some examples, and explain why
  they are safe. What are some common functions that aren't safe, and why?

\begin{quote}
    Some async-signal-safe functions include read(), write(), chmod(), fork(), dup(), and stat().  They are async-signal-safe because they either cannot be interrupted or are rerentrant, meaning that they can safely be simultaneously executed by multiple threads of execution in the same process.  Some non-async-signal-safe functions include malloc(), free(), printf(), and scanf().  This is because they are non-reentrant, which is usually caused by using or updating global or static data structures.
\end{quote}

\item Describe the purpose and usage of the signal mask.

\begin{quote}
    Signal masks are a method of storing 0 or 1 of each type of signal.  They are useful for representing signal sets.
\end{quote}

\item In what ways can a user send a signal to a process?

\begin{quote}
    With kill().
\end{quote}

\item Different functions impact the timestamp on a file in different ways. Give some
  examples, the resulting timestamps that change, and explain why this is.

\begin{quote}

\end{quote}

\item What is the purpose of the \texttt{/proc} filesystem? How is it used? What types of
  information can be found there?

\begin{quote}
    The /proc filesystem is a virtual filesystem that provides an interface to internal kernel data strcutures in the form of regular files and directories.  It can be read from or written to like a normal file systems, and can be used to retrieve various information, including that related to networks, filesystems, devices, and processes.
\end{quote}

\item Explain the idea of locales.

\begin{quote}
    Locales are areas to which a set of display standards apply, such as those for date formats, languages, etc.
\end{quote}

\item What time did the epoch start, locally?

\begin{quote}
16:00 on Dec 31st, 1969
\end{quote}

\item Give a call to \texttt{strftime()} that would replicate the output of
  \texttt{asctime()}.

\begin{quote}
    \texttt{strftime(outstr, maxsize, "\%c", tr)}
\end{quote}

\item What are the different representations of time used by POSIX systems?

\begin{quote}
    struct timeval, time\_t, struct tm, strings
\end{quote}

\item There are several ways to duplicate a file descriptor. Compare and contrast them.

\begin{quote}
\begin{itemize}
    \item dup() gives the first unused file descriptor number.
    \item dup2() gives the requested file descriptor number, closing the old one first if necessary.
    \item dup3() is mostly the same as dup2(), with the major difference being that the caller can force the flag to be set for the new file descriptor by specifying \texttt{O\_CLOEXEC} in \texttt{flags}.
\end{itemize}
\end{quote}

\item \label{last} Describe, in words, the relationship between the per-process file
  descriptor tables, the system wide open file table, and the system-wide inode table.

\begin{quote}

\end{quote}

\end{enumerate}

% \clearpage\null

\end{document}


