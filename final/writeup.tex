\documentclass[letterpaper,10pt,fleqn]{article}

%example of setting the fleqn parameter to the article class -- the below sets the offset from flush left (fl)
\setlength{\mathindent}{1cm}

\usepackage{graphicx}                                        

\usepackage{amssymb}                                         
\usepackage{amsmath}                                         
\usepackage{amsthm}                                          

\usepackage{alltt}                                           
\usepackage{float}
\usepackage{color}

\usepackage{balance}
\usepackage[TABBOTCAP, tight]{subfigure}
\usepackage{enumitem}

\usepackage{pstricks, pst-node}

%the following sets the geometry of the page
\usepackage{geometry}
\geometry{textheight=9in, textwidth=6.5in}

% random comment

\newcommand{\cred}[1]{{\color{red}#1}}
\newcommand{\cblue}[1]{{\color{blue}#1}}

\usepackage{hyperref}

\usepackage{textcomp}
\usepackage{listings}

\usepackage{wasysym}

\usepackage{minted}

\def\name{Sean Rettig}

%% The following metadata will show up in the PDF properties
\hypersetup{
  colorlinks = true,
  urlcolor = black,
  pdfauthor = {\name},
  pdfkeywords = {cs311 ``operating systems''},
  pdftitle = {CS 311 Project},
  pdfsubject = {CS 311 Project},
  pdfpagemode = UseNone
}

\parindent = 0.0 in
\parskip = 0.2 in

\pagestyle{empty}

\numberwithin{equation}{section}

\newcommand{\D}{\mathrm{d}}

\newcommand\invisiblesection[1]{%
  \refstepcounter{section}%
  \addcontentsline{toc}{section}{\protect\numberline{\thesection}#1}%
  \sectionmark{#1}}

\begin{document}

I, \name, hereby state this is my own work, with no help given or received.

%to remove page numbers, set the page style to empty

\section*{CS 311 Final Exam}
\hrule

\subsection*{Differences Between the Windows and POSIX APIs}

\subsubsection*{File I/O}

\subsubsection*{Sockets}

\subsubsection*{derp}

\subsection*{Example of Multi-Platform Program: Simple File Copy Tool}

The following program is also defined in mycopy.c and is bundled with a makefile for compiling on Linux/Unix and an exe for use on Windows (compiled and tested on 64-bit Windows 7 Professional).  Follows C99 standard.

\begin{minted}{c}
#include <stdio.h>
#if defined _WIN32 || defined WIN32 || WIN64 /* Support Windows */
#   include <windows.h>
#   include <tchar.h>
#   include <strsafe.h>

#   ifndef WIN
#       define WIN /* Define convenience variable */
#   endif
#elif defined __unix__ /* Support UNIX/Linux */
#   include <sys/types.h>
#   include <sys/stat.h>
#   include <fcntl.h>
#   include <stdlib.h>
#   include <unistd.h>
#else /* Nothing else is supported */
#   error "Unsupported platform"
#endif

#ifndef BUFSIZE
#   define BUFSIZE 4096
#endif

/* Print usage info and exit */
void usage()
{
#ifdef WIN
    printf("Usage: mycopy.exe SOURCE DEST\n");
#else
    printf("Usage: ./mycopy SOURCE DEST\n");
#endif
    exit(EXIT_FAILURE);
}

/* Print given error message and exit */
void error(char *message)
{
    perror(message);
    exit(EXIT_FAILURE);
}

/* Copies the file at argv[1] to the file at argv[2] in increments of BUFSIZE */
#ifdef WIN
int __cdecl _tmain(int argc, TCHAR *argv[])
#else
int main(int argc, char *argv[])
#endif
{
    if(argc != 3) usage();
    
#ifdef WIN
    HANDLE infile = CreateFile( argv[1],
                                GENERIC_READ,
                                FILE_SHARE_READ,
                                NULL,
                                OPEN_EXISTING,
                                0,
                                NULL);
    HANDLE outfile = CreateFile(argv[2],
                                GENERIC_WRITE,
                                0,
                                NULL,
                                CREATE_ALWAYS,
                                FILE_ATTRIBUTE_NORMAL,
                                NULL);
#else
    int infile = open(argv[1], O_RDONLY);
    int outfile = open(argv[2], O_WRONLY | O_TRUNC | O_CREAT, 0664);
#endif
    
#ifndef NOFILE
#   ifdef WIN
#       define NOFILE INVALID_HANDLE_VALUE
#   else
#       define NOFILE -1
#   endif
#endif

    if(infile == NOFILE)
        error("Could not open input file");
    if(outfile == NOFILE)
        error("Could not open output file");

    char buf[BUFSIZE];

#ifndef BCOUNT
#   ifdef WIN
#       define BCOUNT DWORD
#   else
#       define BCOUNT ssize_t
#   endif
#endif

    BCOUNT bytesread;
    BCOUNT byteswritten;

    for(;;){
#ifdef WIN
        if(ReadFile(infile, buf, BUFSIZE, &bytesread, NULL) == FALSE)
#else
        if((bytesread = read(infile, buf, BUFSIZE)) == -1)
#endif
            error("Problem reading from input file");
        else if(bytesread == 0)
            break;
        else{
#ifdef WIN
            if(WriteFile(outfile, buf, bytesread, &byteswritten, NULL) == FALSE)
#else
            if((byteswritten = write(outfile, buf, bytesread)) == -1)
#endif
                error("Could not write to output file");
            
            if(byteswritten != bytesread)
                error("Could not write whole buffer to output file");
        }
    }

#ifdef WIN
    CloseHandle(infile);
    CloseHandle(outfile);
#else
    close(infile);
    close(outfile);
#endif

    exit(EXIT_SUCCESS);
}
\end{minted}

\end{document}
