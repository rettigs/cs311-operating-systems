\documentclass[letterpaper,10pt,fleqn]{article}

%example of setting the fleqn parameter to the article class -- the below sets the offset from flush left (fl)
\setlength{\mathindent}{1cm}

\usepackage{graphicx}                                        

\usepackage{amssymb}                                         
\usepackage{amsmath}                                         
\usepackage{amsthm}                                          

\usepackage{alltt}                                           
\usepackage{float}
\usepackage{color}

\usepackage{balance}
\usepackage[TABBOTCAP, tight]{subfigure}
\usepackage{enumitem}

\usepackage{pstricks, pst-node}

%the following sets the geometry of the page
\usepackage{geometry}
\geometry{textheight=9in, textwidth=6.5in}

% random comment

\newcommand{\cred}[1]{{\color{red}#1}}
\newcommand{\cblue}[1]{{\color{blue}#1}}

\usepackage{hyperref}

\usepackage{textcomp}
\usepackage{listings}

\usepackage{wasysym}

\usepackage{minted}

\def\name{Sean Rettig}

%% The following metadata will show up in the PDF properties
\hypersetup{
  colorlinks = true,
  urlcolor = black,
  pdfauthor = {\name},
  pdfkeywords = {cs311 ``operating systems''},
  pdftitle = {CS 311 Project},
  pdfsubject = {CS 311 Project},
  pdfpagemode = UseNone
}

\parindent = 0.0 in
\parskip = 0.2 in

\pagestyle{empty}

\numberwithin{equation}{section}

\newcommand{\D}{\mathrm{d}}

\newcommand\invisiblesection[1]{%
  \refstepcounter{section}%
  \addcontentsline{toc}{section}{\protect\numberline{\thesection}#1}%
  \sectionmark{#1}}

\begin{document}

I, \name, hereby state this is my own work, with no help given or received.

%to remove page numbers, set the page style to empty

\section*{CS 311 Final Exam}
\hrule

\subsection*{Differences Between the Windows and POSIX APIs}

\subsubsection*{File I/O}

For basic file I/O, the APIs have nearly identical functionality, and differ mainly in their syntax/usage.  While POSIX specifies the open(), read(), write(), and close() system calls {[}1{]}, the Windows API specifies CreateFile(), ReadFile(), WriteFile(), and CloseHandle() {[}2{]} {[}3{]}.  Also, rather than using file descriptors of an integer type (as in POSIX), Windows uses variables of the HANDLE type to refer to open files.  In addition these differences, it is worth noting that the Windows calls are significantly more verbose than the POSIX ones.

As defined by the Linux man pages, open() simply takes a file path and one or more use flags (which can be OR'd together), as well as a file mode if the file is being created.  Windows' CreateFile() (which is also used to open existing files, despite the name), on the other hand, requires a file path, read/write use flag, sharing mode, security attribute struct, file creation use flag, file attributes and other use/security flags, and a handle to a template file.  Granted, some of these arguments can be NULL or simply not used/initialized, but the verbosity still impedes use compared to open(), which condenses the call by allowing many of these arguments to simply be OR'd together in the second argument if they're needed, or completely unspecified if they're not.  For example, compare the usage of CreateFile() in mycopy.c (embedded further below) compared to open():
\begin{minted}{c}
/* Windows */
CreateFile(argv[1], GENERIC_READ, FILE_SHARE_READ, NULL, OPEN_EXISTING, 0, NULL);

/* POSIX */
open(argv[1], O_RDONLY);
\end{minted}
Both accomplish the same task, but CreateFile() is three times as long.  The increased verbosity is a recurring pattern through much of the Windows API; read() and write() simply take a file descriptor/handle, data buffer, and number of bytes to read/write, but ReadFile() and WriteFile() also require a pointer to a variable storing the number of bytes actually read/written, as well as an OVERLAPPED struct if the file was opened with the FILE\_FLAG\_OVERLAPPED flag (though it can be specified as NULL otherwise).  Additionally, ReadFile() and WriteFile() only return a boolean value, so error checking must be performed on two different variables (the return value as well as the number of bytes read/written) to make sure that the call not only did not terminate with an error, but also read/wrote the correct number of bytes.  The read() and write() calls, on the other hand, return the number of bytes read/written, or -1 on an error.  This is a minor simplification, but a welcome one in a situation where there is already plenty of complexity to consider.  For example, consider checking to see if we are at the end of a file (code from mycopy.c):
\begin{minted}{c}
/* Windows */
if(ReadFile(infile, buf, BUFSIZE, &bytesread, NULL) == TRUE && bytesread == 0)

/* POSIX */
if(read(infile, buf, BUFSIZE) == 0)
\end{minted}
Not only is the ReadFile() function physically longer, its return value must be checked to be TRUE (in addition to checking that 0 bytes were read) since bytesread will also be 0 if the function terminates with an error.  The read() call need only check if the number of bytes read was 0.

Writing is similar to reading (code from mycopy.c):
\begin{minted}{c}
/* Windows */
WriteFile(outfile, buf, bytestowrite, &byteswritten, NULL)

/* POSIX */
byteswritten = write(outfile, buf, bytestowrite)
\end{minted}

The rare case where the corresponding calls are nearly identical is with closing open files; both close() and closeHandle() both simply take the file descriptor/handle as the sole argument (code from mycopy.c):
\begin{minted}{c}
/* Windows */
CloseHandle(infile);

/* POSIX */
close(infile);
\end{minted}

\subsubsection*{Sockets}
The sockets APIs for POSIX and Windows are actually surprisingly similar compared to how different they were for file I/O, the main difference being that rather than integer file descriptors or even HANDLEs, the Windows Winsock2 API functions use variables of the SOCKET type.  POSIX specifies the socket(), bind(), listen(), accept(), connect(), and close() functions {[}1{]}for performing basic socket operations, while the Winsock API specifies the same ones (save for close(), which is called closesocket() in Winsock) {[}4{]}.

These functions all take similar arguments and have similar return values; for example, both versions of socket() take a domain (such as AF\_INET), a type (such as SOCK\_STREAM), and a protocol (e.g. TCP/IPPROTO\_TCP, protocol number 6).  Since most common types only have one protocol, both functions support leaving the protocol argument at 0.  Both functions also return a reference to the socket (POSIX as an integer file descriptor, Winsock as a SOCKET), or -1 on error.  Example usage of socket code (code from server\_windows.c, client\_windows.c, server\_posix.c, and client\_posix.c):
\begin{minted}{c}
/* Windows server */
WSADATA wsaData;
WSAStartup(MAKEWORD(2, 2), &wsaData);

SOCKET listens = socket(AF_INET, SOCK_STREAM, IPPROTO_TCP);
bind(listens, (struct sockaddr *) &srvaddr, addrlen);
listen(listens, BACKLOG);
SOCKET socks = accept(listens, (struct sockaddr *) &cliaddr, &addrlen);
closesocket(socks);
closesocket(listens);

/* Windows client */
WSADATA wsaData;
WSAStartup(MAKEWORD(2, 2), &wsaData);

SOCKET socks = socket(AF_INET, SOCK_STREAM, IPPROTO_TCP);
connect(socks, (struct sockaddr *) &srvaddr, addrlen);
closesocket(socks);

/* POSIX server*/
int listenfd = socket(AF_INET, SOCK_STREAM, 0);
bind(listenfd, (struct sockaddr *) &srvaddr, addrlen);
listen(listenfd, BACKLOG);
int sockfd = accept(listenfd, (struct sockaddr *) &cliaddr, &addrlen);
close(listenfd);
close(sockfd);

/* POSIX client */
int sockfd = socket(AF_INET, SOCK_STREAM, 0);
connect(sockfd, (struct sockaddr *) &srvaddr, addrlen);
close(sockfd);
\end{minted}
The only real difference between the POSIX and Windows APIs here is that Windows needs to "initialize" Winsock before any connections can be made, though this difference is fairly trivial.  It is worth noting, however, that by default, the standard file reading and writing functions on Windows do not work with SOCKETs; rather, recv() and send() {[}4{]} should be used.  In terms of syntax and usage, they more closely resemble read() and write() from the POSIX API than ReadFile() and WriteFile() from the Windows API (code from server\_windows.c and server\_posix.c):
\begin{minted}{c}
/* Windows */
bytesread = recv(socks, buf, BUFSIZE, 0);
byteswritten = send(socks, buf, bytesread, 0);

/* POSIX */
bytesread = read(sockfd, buf, BUFSIZE);
byteswritten = write(sockfd, buf, bytesread);
\end{minted}
As shown above, the usage of recv() and read() (as well as send() and write()) are almost identical, with the only difference being that recv() and send() take an additional 4th argument for use flags for how the data should be received or sent (e.g. the MSG\_PEEK flag for recv(), which reads from the socket without modifying what's in its buffer).

\subsubsection*{derp}

\subsection*{Example of Multi-Platform Program: Simple File Copy Tool}

The following program is also defined in mycopy.c and is bundled with a makefile for compiling on Linux/Unix and an exe for use on Windows (compiled and tested on 64-bit Windows 7 Professional).

\begin{minted}{c}
#include <stdio.h>
#if defined _WIN32 || defined WIN32 || WIN64 /* Support Windows */
#   include <windows.h>
#   include <tchar.h>
#   include <strsafe.h>

#   ifndef WIN
#       define WIN /* Convenience variable */
#   endif
#   ifndef BCOUNT
#       define BCOUNT DWORD /* How to store byte count */
#   endif
#   ifndef FTYPE
#       define FTYPE HANDLE /* How to store open file references */
#   endif
#elif defined __unix__ /* Support UNIX/Linux */
#   include <sys/types.h>
#   include <sys/stat.h>
#   include <fcntl.h>
#   include <stdlib.h>
#   include <unistd.h>

#   ifndef BCOUNT
#       define BCOUNT ssize_t /* How to store byte count */
#   endif
#   ifndef FTYPE
#       define FTYPE int /* How to store open file references */
#   endif
#else /* Nothing else is supported */
#   error "Unsupported platform"
#endif
#ifndef BUFSIZE
#   define BUFSIZE 4096
#endif

/* Print usage info and exit */
void usage()
{
#ifdef WIN
    printf("Usage: mycopy.exe SOURCE DEST\n");
#else
    printf("Usage: ./mycopy SOURCE DEST\n");
#endif
    exit(EXIT_FAILURE);
}

/* Print given error message and exit */
void error(char *message)
{
    perror(message);
    exit(EXIT_FAILURE);
}

/* Copy the file at argv[1] to the file at argv[2] in increments of BUFSIZE */
#ifdef WIN
int __cdecl _tmain(int argc, TCHAR *argv[])
#else
int main(int argc, char *argv[])
#endif
{
    FTYPE infile, outfile;
    BCOUNT bytesread, byteswritten;
    char buf[BUFSIZE];

    if(argc != 3) usage();

#ifdef WIN
    infile = CreateFile (argv[1], GENERIC_READ, FILE_SHARE_READ,
                         NULL, OPEN_EXISTING, 0, NULL);
    outfile = CreateFile(argv[2], GENERIC_WRITE, 0,
                         NULL, CREATE_ALWAYS, FILE_ATTRIBUTE_NORMAL, NULL);
#else
    infile = open(argv[1], O_RDONLY);
    outfile = open(argv[2], O_WRONLY | O_TRUNC | O_CREAT, 0664);
#endif
    
#ifndef NOFILE
#   ifdef WIN
#       define NOFILE INVALID_HANDLE_VALUE
#   else
#       define NOFILE -1
#   endif
#endif

    if(infile == NOFILE) error("Could not open input file");
    if(outfile == NOFILE) error("Could not open output file");

    for(;;){
#ifdef WIN
        if(ReadFile(infile, buf, BUFSIZE, &bytesread, NULL) == FALSE)
#else
        if((bytesread = read(infile, buf, BUFSIZE)) == -1)
#endif
            error("Problem reading from input file");
        else if(bytesread == 0)
            break;
        else{
#ifdef WIN
            if(WriteFile(outfile, buf, bytesread, &byteswritten, NULL) == FALSE)
#else
            if((byteswritten = write(outfile, buf, bytesread)) == -1)
#endif
                error("Could not write to output file");
            if(byteswritten != bytesread)
                error("Could not write whole buffer to output file");
        }
    }

#ifdef WIN
    CloseHandle(infile);
    CloseHandle(outfile);
#else
    close(infile);
    close(outfile);
#endif

    exit(EXIT_SUCCESS);
}
\end{minted}

\subsection*{References}

\begin{itemize}
    \item[{[}1{]}] Linux man pages: http://man7.org/linux/man-pages/
    \item[{[}2{]}] MSDN Library - File Management Functions: http://msdn.microsoft.com/en-us/library/aa364232.aspx
    \item[{[}3{]}] MSDN Library - Handle and Object Functions: http://msdn.microsoft.com/en-us/library/ms724461.aspx
    \item[{[}4{]}] MSDN Library - Winsock Functions: http://msdn.microsoft.com/en-us/library/windows/desktop/ms741394.aspx
    \item[{[}5{]}] MSDN Library - : 
\end{itemize}

\end{document}
