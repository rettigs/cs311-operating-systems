\documentclass[letterpaper,10pt,fleqn]{article}

%example of setting the fleqn parameter to the article class -- the below sets the offset from flush left (fl)
\setlength{\mathindent}{1cm}

\usepackage{graphicx}                                        

\usepackage{amssymb}                                         
\usepackage{amsmath}                                         
\usepackage{amsthm}                                          

\usepackage{alltt}                                           
\usepackage{float}
\usepackage{color}

\usepackage{balance}
\usepackage[TABBOTCAP, tight]{subfigure}
\usepackage{enumitem}

\usepackage{pstricks, pst-node}

%the following sets the geometry of the page
\usepackage{geometry}
\geometry{textheight=9in, textwidth=6.5in}

% random comment

\newcommand{\cred}[1]{{\color{red}#1}}
\newcommand{\cblue}[1]{{\color{blue}#1}}

\usepackage{hyperref}

\usepackage{textcomp}
\usepackage{listings}

\usepackage{wasysym}

\def\name{Sean Rettig}

%% The following metadata will show up in the PDF properties
\hypersetup{
  colorlinks = true,
  urlcolor = black,
  pdfauthor = {\name},
  pdfkeywords = {cs311 ``operating systems''},
  pdftitle = {CS 311 Project},
  pdfsubject = {CS 311 Project},
  pdfpagemode = UseNone
}

\parindent = 0.0 in
\parskip = 0.2 in

\pagestyle{empty}

\numberwithin{equation}{section}

\newcommand{\D}{\mathrm{d}}

\newcommand\invisiblesection[1]{%
  \refstepcounter{section}%
  \addcontentsline{toc}{section}{\protect\numberline{\thesection}#1}%
  \sectionmark{#1}}

\begin{document}

%to remove page numbers, set the page style to empty

\section*{Assignment 2 Write-up}
\hrule

\subsection*{Design}
\begin{itemize}
    \item Bdiff will open both files for reading and will read in bytes in chunks of 4096 (for performance reasons), compare them, output the results, and repeat until one of the files ends.
    \item Bdiff -b will simply print both the octal value and character representation of each byte.
    \item Bdiff -i will simply offset where it starts reading each file.
    \item Bdiff -n will simply stop comparing early once it has compared the given number of bytes.
    \item Bdiff -s will simply not print anything, only return an exit status.  For perforamnce reasons, it will stop comparing once it has found at least one difference.
    \item Bdiff -r will open the directory streams using the opendir() system call and compare them as normal.
    \item Bpatch \[-r\] will read the output of bdiff and fill a buffer of 4096 bytes with patch data until it finds a non-sequential run or until the buffer fills up, at which point it will write all changes to the file/directory at once and then start filling another buffer (until all of the bdiff output has been processed).
\end{itemize}

\subsection*{Work Log}

\subsection*{Challenges Overcame}
\begin{itemize}
    \item Does bdiff need to accept human-readable numbers for -n?
    \item Does bpatch need to accept the output of bdiff -b?
\end{itemize}

\subsection*{Questions}
\begin{enumerate}
    \item The main point of the assignment was to learn the basics of performing file I/O in C using the Linux programming interface.
    \item 
\end{enumerate}

\end{document}
